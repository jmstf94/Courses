\newpage
\thispagestyle{plain}

\vspace*{\fill}
\begin{center}
    \vspace{0.9cm}
    \textbf{\en{Abstract}}
\end{center}
{\en The scope of this thesis is random simulations and in particular random walk experiments. We start with some basic theory on random walks regarding mean square displacement, number of distinct sites that the random walker visits and the Rosenstock approximation. After that, we begin with a random walk on one and two dimensions calculating the mean square displacement experimentally. Then, we do a continuous random walk while calculating again the mean square displacement. The third experiment calculates the number of distinct sites the random walker has visited in one and two dimension. At last, we run an experiment in a grid with molecule traps and compare our results with the Rosenstock Approximation in order to check its validity.}
\vspace*{\fill}
\newpage
\vspace*{\fill}
\begin{center}
    \vspace{0.9cm}
    \textbf{\gr{Περίληψη}}
\begin{otherlanguage}{greek}

Ο στόχος της εργασίας είναι οι στοχαστικές προσομοιώσεις και συγκεκριμένα τα πειράματα τυχαίων περιπάτων. Αρχικά αναφέρουμε την βασική θεωρία γύρω από τους τυχαίους περιπάτους όσον αφορά την μέση τετραγωνική μετατόπιση, τον αριθμό των διαφορετικών θέσεων που επισκέφτηκε το σωματίδιο και την προσέγγιση {\en Rosenstock}. Στη συνέχεια, ξεκινάμε με ένα τυχαίο περίπατο σε μία και δύο διαστάσεις υπολογίζοντας την μέση τετραγωνική μετατόπιση. Παρακάτω, επαναλαμβάνουμε τον υπολογισμό αλλά αυτή τη φορά σε δισδιάστατο συνεχή χώρο. Το τρίτο πείραμα υπολογίζει τον αριθμό των διαφορετικών θέσεων που επισκέπτεται ο περιπατητής σε μία και δύο διαστάσεις. Καταλήγουμε υπολογίζοντας την πυκνότητα των χρόνων παγίδευσης σε ένα πλέγμα με μόρια-παγίδες και συγκρίνοντας τα αποτελέσματά μας με την θεωρητική προσέγγιση {\en Rosenstock}.
\end{otherlanguage}
\end{center}
\begin{otherlanguage}{greek}
\end{otherlanguage}
\vspace*{\fill}