\chapter{Συμπεράσματα}
\begin{itemize}
\item Από το \ref{Part1_a} βρίσκουμε το αναμενόμενο από την θεωρία μας. Ο μέσος όρος των τετραγωνικών μετατοπίσεων των 100000 μονοδιάστατων περιπάτων, είναι περίπου ίσος με τον αριθμό των βημάτων του κάθε περιπάτου, εφόσον έχουμε βήμα ίσο με 1.
\item Από το \ref{Part1_b} βρίσκουμε το αναμενόμενο από την θεωρία μας.  Ο μέσος όρος των τετραγωνικών μετατοπίσεων των 100000 δισδιάστατων περιπάτων, είναι περίπου ίσος με τον αριθμό των βημάτων του κάθε περιπάτου, εφόσον έχουμε βήμα ίσο με 1.
\item Από το \ref{CRW} βρίσκουμε ότι ανεξάρτητα από τον τρόπο που διαμερίζουμε τον χώρο, ο μέσος όρος των τετραγωνικών μετατοπίσεων παραμένει περίπου ίσος του αριθμού των βημάτων. Επίσης, από τον σχεδιασμό της ευθείας ελαχίστων τετραγώνων, παρατηρούμε ότι τα οι μέσες τετραγωνικές μετατοπίσεις είναι συνεπής με τη θεωρία μας, είδη από τα 100 βήματα. Η σχέση των βημάτων με τις τετραγωνικές μετατοπίσεις φαίνεται να είναι γραμμικές, μιας και τα {\en error} της ευθείας ελαχίστων τετραγώνων φαίνονται εξαιρετικά μικρά. 
\item
Από το \ref{S1D} παρατηρούμε ότι η καμπύλη $S_n$ συναρτήσει του $n$ που σχεδιάσαμε (έστω και ανά 100 βήματα) είναι σε σύμβαση με τη θεωρητική της μορφή \eqref{1D_S}.
\item
Από το \ref{S2D} παρατηρούμε ότι η καμπύλη $S_n$ συναρτήσει του $n$ που σχεδιάσαμε (έστω και ανά 100 βήματα) είναι σε σύμβαση με τη θεωρητική της μορφή \eqref{2D_S}.
\item Από το \ref{rosen} παρατηρούμε ότι από τα 1000 περίπου βήματα και μετά, η θεωρητική προσέγγιση {\en Rosenstock} αναπαράγει πολύ καλά τα πειραματικά μας αποτελέσματα. Εφόσον ο τύπος της κατανομής προκύπτει από πιθανοκρατικές υποθέσεις και υπολογισμό μέσων όρων είναι λογικό. Σύμφωνα με το νόμο τον μεγάλων αριθμών, όσο περισσότερα βήματα κάνω τόσο καλύτερα θα προσεγγίζονται οι στατιστικές μου προβλέψεις.
\end{itemize}