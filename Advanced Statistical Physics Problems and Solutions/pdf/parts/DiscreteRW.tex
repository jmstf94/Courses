\section{Διακριτός Τυχαίος Περίπατος - Μέση τετραγωνική μετατόπιση}
\subsection{Σε μία διάσταση}
\label{Part1_a}
Η πρώτη και απλούστερη προσομοίωση που θα κάνουμε, αφορά ένα τυχαίο περίπατο σωματιδίου σε ένα πλέγμα μίας διάστασης. Ο σκοπός είναι να υπολογίσουμε την μέση τετραγωνική μετατόπιση 100000 προσομοιώσεων που ονομάζουμε {\en runs}, όπως φαίνεται και στον κώδικα. Το κάθε {\en run} θα κάνει {\en t}=1000 τυχαία βήματα είτε αριστερά είτε δεξιά μεγέθους {\en step}=1.

Για να μοντελοποιήσουμε έναν τέτοιο τυχαίο περίπατο, φανταζόμαστε την γραμμή των ακεραίων αριθμών και επιλέγουμε αυθαίρετα ένα σημείο εκκίνησης. Για απλότητα, επιλέγουμε ως αρχική θέση την αρχή του άξονα δηλαδή {\en position}=0. Στη συνέχεια ένας βρόγχος επανάληψης {\en t}=1000 βημάτων ανανεώνει την θέση του σωματιδίου προσθέτοντας ή αφαιρώντας {\en step}=1 στη μεταβλητή {\en position}. Η τυχαία επιλογή του βήματος αριστερά ή δεξιά γίνεται με την εντολή {\en random.choice} που επιλέγει με ίση πιθανότητα ένα μέλος της λίστας {\en [-step,+step]}. Έτσι το βήμα αριστερά γίνεται αν στην θέση που βρίσκεται το σωματίδιο προσθέσουμε -1, ενώ το βήμα δεξιά αν προσθέσουμε +1. Επομένως, ο αλγόριθμος ενός πειράματος έχει ως εξής:
\en

\pagebreak

\begin{lstlisting}
//=============================================================//
set starting position 

For t=1000 steps:
    choose randomly(uniformly) if the particle goes left or right
    update particle position
//=============================================================//
\end{lstlisting}
\gr
H ομοιόμορφη κατανομή εξασφαλίζει ότι σε κάθε βήμα το σωματίδιο έχει $1/2$ πιθανότητα να μεταβεί στην θέση αριστερά του, και $1/2$ πιθανότητα να μεταβεί στη θέση δεξιά του. O κώδικας που υλοποιεί τον παραπάνω αλγόριθμο:
\en
\begin{python}
position = 0
for j in range(t):
    move = random.choice([-step,+step])
    position = position + move
\end{python}

\gr
Έχοντας λοιπόν την τελική θέση είναι εύκολο να υπολογίσουμε την τετραγωνική μετατόπιση του πειράματος, απλά υψώνοντας την μεταβλητή {\en position} στο τετράγωνο. Σκοπός όμως είναι να τρέξουμε το πείραμα 100000 φορές και να υπολογίσουμε την μέση τετραγωνική απόσταση όλων των {\en runs}. Άρα στον τελικό κώδικα αρχικοποιούμε μία μεταβλητή {\en \texttt{sd\_sum}}=0 (ακρωνύμιο του {\en square distance sum}), και σε κάθε {\en run} προσθέτουμε την τελική τετραγωνική μετατόπιση. Έτσι, υπολογίζουμε το άθροισμα όλων των τετραγωνικών μετατοπίσεων για τα 100000 πειράματα. Τέλος, διαιρούμε αυτό τον αριθμό με τον αριθμό των πειραμάτων που τρέξαμε προκειμένου να βρούμε την μέση τετραγωνική μετατόπιση {\en \texttt{meam\_sd}}.


\en
\begin{python}
start_time = time.time()

sd_sum = 0
runs = 100000
t=1000
step = 1
for i in range(runs):
    position = 0
    for j in range(t):
        move = random.choice([-step,+step])
        position = position + move
    sd_sum+=position**2
mean_sd = sd_sum/runs


print(f"The Mean Square Displacement is {mean_sd}.")
print(f"Execution Time: {(time.time() - start_time)} seconds.")
\end{python}
\gr 
και το {\en output}:
\en
\begin{python}
The Mean Square Displacement is 995.37292.
Execution Time: 55.54194784164429 seconds.
\end{python}
\gr
\subsection{Σε δύο διαστάσεις}
\label{Part1_b}
Την ίδια λογική θα χρησιμοποιήσουμε στο πείραμα δύο διαστάσεων. Αυτή την φορά αντί να έχουμε μία γραμμή ακεραίων φανταζόμαστε δύο, κάθετες μεταξύ τους, με σημείο τομής το (0,0) όπως σε ένα καρτεσιανό σύστημα αξόνων. Η διαφορά είναι ότι ο χώρος μας αποτελείται από τα διακριτά ζεύγη των ακεραίων $(x,y) \in \mathbb{Z} \times \mathbb{Z} $. Έχουμε δηλαδή ένα πλέγμα δύο διαστάσεων πάνω στο οποίο το σωματίδιό μας θα εκτελέσει τον τυχαίο περίπατο. Η αρχική μας θέση πλέον έχει δύο συνιστώσες, την τετμημένη και την τεταγμένη. 

Aρχικοποιούμε για το ένα {\en run} δύο μεταβλητές {\en \texttt{position\_x}}=0 και {\en \texttt{position\_y}}=0. Στη συνέχεια για την μία εκτέλεση του πειράματος {\en t}=1000 βημάτων, επιλέγουμε σε κάθε επανάληψη τυχαία, αν το σωματίδιο θα κινηθεί αριστερά, δεξιά, πάνω ή κάτω. Ανανεώνουμε την τετμημένη και την τεταγμένη αναλόγως, και έχουμε πλέον ως νέα θέση του σωματιδίου στο δισδιάστατο πλέγμα τις νέες τιμές
{\en (\texttt{position\_x}, \texttt{position\_y})}. Συνεπώς, ο αλγόριθμος ενός πειράματος έχει ως εξής:
\en
\begin{lstlisting}
//=============================================================//
set starting position 
For t=1000 steps:
    choose randomly(uniformly) if the particle goes 
                            left, right, up or down
    update particle position
//=============================================================//
\end{lstlisting}
\gr

Προφανώς η ομοιόμορφη τυχαία επιλογή μετάβασης δίνει $1/4$ πιθανότητα για την κάθε κατεύθυνση που μπορεί να κινηθεί το σωματίδιο. 
Η μετάβαση αριστερά σημαίνει πρόσθεση -1 στην τετμημένη και 0 στην τεταγμένη, η μετάβαση επάνω πρόσθεση +1 στην τεταγμένη και 0 στην τετμημένη κ.ο.κ.  Ο κώδικας που υλοποιεί το ένα πείραμα είναι:  
\newpage
\en
\begin{python}
position_x = 0
position_y = 0
for j in range(t):
    move = random.choice([[-step,0],[step,0],[0,-step],[0,step]])
    position_x+= move[0]
    position_y+= move[1]
\end{python}
\gr 
Έχοντας την τελική θέση του σωματιδίου μπορούμε να υπολογίσουμε την τετραγωνική μετατόπιση με χρήση του πυθαγόρειου θεωρήματος ως $d^2 = x^2+y^2$.

Τρέχουμε λοιπόν το πρόγραμμα 100000 φορές και κάθε φορά προσθέτουμε την τετραγωνική μετατόπιση στην μεταβλητή  {\en \texttt{sd\_sum}}:
\en 
\begin{python}
sd_sum+=position_x**2+position_y**2
\end{python}
\gr
Φυσικά στο τέλος διαιρούμε με τον αριθμό των πειραμάτων που εκτελέσαμε για να βρούμε την μέση τετραγωνική μετατόπιση των 100000 {\en runs}. Ο συνολικός κώδικας είναι ως εξής \en
\begin{python}
start_time = time.time()

sd_sum = 0
runs = 100000
t=1000
step = 1
for i in range(runs):
    position_x = 0
    position_y = 0
    for j in range(t):
        move = random.choice([[-step,0],[step,0],[0,-step],[0,step]])
        position_x+= move[0]
        position_y+= move[1]
    sd_sum+=position_x**2+position_y**2
mean_sd = sd_sum/runs

print(f"The Mean Square Displacement is {mean_sd}.")
print(f"Execution Time: {(time.time() - start_time)} seconds.")
\end{python}
\gr 
με \en output:
\begin{python}
The Mean Square Displacement is 1009.55048.
Execution Time: 79.88995146751404 seconds.
\end{python}
\gr    