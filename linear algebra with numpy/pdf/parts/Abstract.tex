\thispagestyle{plain}
\newpage
\vspace*{\fill}
\begin{center}
    \vspace{0.9cm}
    \textbf{\gr{Περίληψη}}
\end{center}
\begin{otherlanguage}{greek}
Σε αυτή την εργασία μελετάμε βασικά θέματα Γραμμικής Άλγεβρας μέσω της βιβλιοθήκης της {\en Python}, {\en numpy}. Στην αρχή, αναφέρουμε τα βασικά αντικείμενα(πίνακες, διανύσματα) και τις κύριες πράξεις(εσωτερικό γινόμενο, πολλαπλασιασμός πινάκων κλπ) στη γραμμική άλγεβρα. Στη συνέχεια, συνδέουμε τους πίνακες με την επίλυση γραμμικού συστήματος, και αναφέρουμε βασικούς αριθμητικούς αλγόριθμους όπως η απαλοιφή {\en Gauss} και η μέθοδος {\en Jacobi}. Παρακάτω, αναλύουμε τις διαφορετικές παραγοντοποιήσεις και την εύρεσή τους μέσω έτοιμων συναρτήσεων. Τέλος, συζητάμε για τη γραμμική μέθοδο ελαχίστων τετραγώνων ενώ έχουμε δώσει ένα απλό παράδειγμα για τις μεθόδους συμπίεσης εικόνας μέσω της αποσύνθεσης σε ιδιάζουσες τιμές.
\end{otherlanguage}
\vspace*{\fill}