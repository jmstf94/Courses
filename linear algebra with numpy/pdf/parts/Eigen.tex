\chapter{Ιδιοτιμές, ιδιοδιανύσματα και διαγωνιοποίηση}
Τα ιδιοδιανύσματα είναι διανύσματα για τα οποία ισχύει η παρακάτω εξίσωση:
\begin{equation}
A \mathbf{u}=\lambda \mathbf{u}
\label{eigeneq}
\end{equation}
όπου Α είναι ένας τετραγωνικός πίνακας. Για να βρούμε τις ιδιοτιμές $\lambda$ λύνουμε την εξίσωση:
\begin{equation}
|A-\lambda I|=\left(\lambda_1-\lambda\right)\left(\lambda_2-\lambda\right) \cdots\left(\lambda_n-\lambda\right)=0
\end{equation}
Στη συνέχεια μελετάμε την τη αλγεβρική πολλαπλότητα, που αναφέρεται στην περίπτωση όπου μία εκ των ιδιοτιμών είναι πολλαπλή λύση στο παραπάνω πολυώνυμο. Η γεωμετρική πολλαπλότητα έχει να κάνει με τον διανυσματικό χώρο που σχηματίζουν τα ιδιοδιανύσματα των ιδιοτιμών. Στη συνέχεια υπολογίζουμε τα ιδιοδιανύσματα  με την χρήση της \eqref{eigeneq} και βρίσκουμε την γεωμετρική πολλαπλότητα της κάθε ιδιοτιμής. Αν βρούμε $n$ γραμμικώς ανεξάρτητα διανύσματα
 τότε ο πίνακάς μας είναι και διαγωνιοποιήσιμος. Αυτό σημαίνει ότι υπάρχει πίνακας $P$ για τον οποίο:
\begin{equation}
P^{-1} A P=\left[\begin{array}{cccc}
\lambda_1 & 0 & \cdots & 0 \\
0 & \lambda_2 & \cdots & 0 \\
\vdots & \vdots & \ddots & \vdots \\
0 & 0 & \cdots & \lambda_n
\end{array}\right]
\label{eigendec}
\end{equation}
και αυτός θα είναι ο πίνακας των ιδιοδιανυσμάτων τοποθετημένα κατακόρυφα από αριστερά προς τα δεξιά. Ουσιαστικά παραπάνω έχουμε την διαγωνιοποιημένη μορφή του $A$. Αν πολλαπλασιάσουμε από αριστερά με τον πίνακα $P$ και από δεξιά με τον πίνακα $P$ παίρνουμε την ιδιοπαραγοντοποίηση του συστήματος. Πάμε να δούμε πως τα παραπάνω υλοποιούνται με την {\en numpy}. 
\en
\begin{python}
import numpy as np
import numpy.linalg as lin

# define example matrix
A = np.array([
[1, 2, 3],
[4, 5, 6],
[7, 8, 9]])

# factorize
###calculate eigenvalues and the corresponding eigenvectors###
values, vectors = lin.eig(A)

######we create the 
P = vectors.T
D = np.diag(values)

print(f'The eigenvalues per column:{values}')
print(f'The corresponding eigenvectors:\n{P}')
\end{python}
\vspace*{-0.7cm}
\begin{codeout}
The eigenvalues:[ 1.61168440e+01 -1.11684397e+00 -3.38433605e-16]
The corresponding eigenvectors:
[[-0.23197069 -0.52532209 -0.8186735 ]
 [-0.78583024 -0.08675134  0.61232756]
 [ 0.40824829 -0.81649658  0.40824829]]
\end{codeout}
\gr
Εδώ παρατηρούμε κατά τα γνωστά ότι η {\en numpy} χρησιμοποιεί αριθμητικές προσεγγιστικές μεθόδους για τον υπολογισμό των ιδιοτιμών και των ιδιοδιανυσμάτων. Στη συνέχει ελέγχουμε αν όντως ισχύει ο η εξίσωση \eqref{eigendec} με χρήση του τελεστή $@$ που εκτελεί πολλαπλασιασμό πινάκων όμοιο με την {\en np.matmul}:
\en
\begin{python}
print(lin.inv(P)@D@P)
\end{python}
\vspace*{-0.7cm}
\begin{codeout}
[[1. 4. 7.]
 [2. 5. 8.]
 [3. 6. 9.]]
\end{codeout}
\gr

Καταφέραμε λοιπόν να ιδιοπαραγοντοποιήσουμε τον πίνακα $Α$. Οι παραγοντοποιήσεις είναι γενικά πολύ χρήσιμες στην γραμμική άλγεβρα. Θα δούμε παρακάτω επιπλέον τρόπους να παραγοντοποιούμε πίνακες μαζί με κάποιες εφαρμογές τους.