\chapter{Απαλοιφή {\en Gauss}}
Μία από τις ακριβείς αριθμητικές μεθόδους επίλυσης γραμμικών συστημάτων είναι η Απαλοιφή {\en Gauss}. Η απαλοιφή {\en Gauss} βασίζεται στην μετατροπή του επαυξημένου πίνακα σε κλιμακωτή ή ανηγμένη κλιμακωτή μορφή. Πριν ορίσουμε όλα τα παραπάνω, ας μιλήσουμε για τις πράξεις που μπορούμε να κάνουμε μέσα σε ένα πίνακα ενός συστήματος χωρίς να επηρεάζει το αποτέλεσμα.
\begin{itemize}
\item Μπορούμε να προσθαφαιρέσουμε γραμμές μεταξύ τους.
\item Μπορούμε να πολλαπλασιάσουμε μία γραμμή με έναν αριθμό.
\item Μπορούμε να αλλάξουμε θέση δύο γραμμές ή δύο στήλες (εκτός της τελευταίας του επαυξημένου). 
\end{itemize}
Αν εκτελεστούν αυτές τις πράξεις σε ένα σύστημα και στη συνέχεια στον πίνακα του συστήματος, είναι εύκολο να δει κανείς γιατί δεν επηρεάζουν την λύση του συστήματος. Είναι σαν να εκτελούμε γραμμικές πράξεις μεταξύ εξισώσεων.

Ο επαυξημένος πίνακας ενός συστήματος είναι o πίνακας της μορφής:
\begin{equation}
(\mathbf{A} \mid \mathbf{b})=\left[\begin{array}{lll|l}
a_{11} & \cdots & a_{1n} & b_{1} \\
\vdots & \ddots & \vdots & \vdots \\
a_{n1} & \cdots & a_{nn} & b_{n}
\end{array}\right]
\end{equation}
Όπου προφανώς έχουμε τετραγωνικό πίνακα αριστερά (λόγω μεταβλητών και εξισώσεων) και το διάνυσμα $b$ από δεξιά. Μπορούμε λοιπόν να λύσουμε το σύστημα με τρεις τρόπους, ανάγοντας με πράξεις γραμμών και στηλών τον επαυξημένο πίνακα σε τρεις διαφορετικούς πίνακες:
\begin{itemize}
\item Κλιμακωτή μορφή. 
\item Ανηγμένη κλιμακωτή μορφή.
\item Μοναδιαίος πίνακας.
\end{itemize}
Η κλιμακωτή μορφή είναι αυτή που θα κατασκευάσουμε με τον αλγόριθμο της απαλοιφής {\en Gauss}. Στην πράξη δεν χρειαζόμαστε κάποιον αλγόριθμο απαραίτητα γιατί μπορούμε να σκεφτούμε τις κατάλληλες γραμμοπράξεις για να έχουμε το γρήγορο επιθυμητό αποτέλεσμα. Κλιμακωτός πίνακας είναι ο επαυξημένος πίνακας όταν αριστερά της γραμμής διαχωρισμού (δηλαδή στην μεριά του πίνακα A) υπάρχει ένας άνω τριγωνικός πίνακας. Ο ανηγμένος κλιμακωτός αναφέρεται όταν ο πίνακας αυτός είναι άνω τριγωνικός αλλά έχει και τα διαγώνια στοιχεία του ίσα με 1. Είναι προφανές ότι αν σχηματίσουμε αριστερά της διαχωριστικής γραμμής τον μοναδιαίο πίνακα ουσιαστικά έχουμε λύσει το σύστημα. 

Ο αλγόριθμος θα πρέπει να εφαρμοσθεί σε πίνακα με μη μηδενικά διαγώνια στοιχεία. Για να φέρουμε τον πίνακα σε ανηγμένη κλιμακωτή μορφή:
\en 
\begin{codeout}
Gauss Elimination on Matrix A:

For i = 1 to n-1
    For j = i+1 to n
         Ratio = Aj,i/Ai,i
         For k = 1 to n+1
              Aj,k = Aj,k - Ratio * Ai,k
         Next k
    Next j
Next i
\end{codeout}
\gr 
Για να βρούμε την λύση του συστήματος αντικαθιστούμε από κάτω προς τα πάνω τις μεταβλητές μας.
\en
\begin{codeout}
Obtaining Solution by Back Substitution:

Xn = An,n+1/An,n
For i = n-1 to 1 (Step: -1)
     Xi = Ai,n+1
     For j = i+1 to n
         Xi = Xi - Ai,j * Xj
     Next j
     Xi = Xi/Ai,i
Next i
\end{codeout}
\gr
Ας χρησιμοποιήσουμε τον παραπάνω αλγόριθμο χειροκίνητα για να λύσουμε το παρακάτω σύστημα βήμα-βήμα με χρήση της {\en numpy}:
\begin{equation}
\left\{\begin{array}{l}
1 x_1+2 x_2-3x_3=7 \\
3 x_1- x_2+4x_3=6 \\
-2 x_1+ x_2+x_3=1
\end{array}\right.
\end{equation}
Κατασκευάζουμε τον επαυξημένο πίνακα του συστήματος. Η εντολή {\en concatenate} μπορεί να ενώσει δύο {\en np.array}, είτε βάζοντας το δεύτερο {\en np.array} από κάτω ({\en axis = 0}), είτε βάζοντας το δεύτερο  {\en np.array} από δεξιά ({\en axis = 1})
\en
\begin{python}
import numpy as np

######define the system######
A = np.array([[1,2,-3],[3,-1,4],[-2,1,1]])
b = np.array([[7,6,1]])
######define the augmented matrix#######
augmat = np.concatenate((A,b.T),axis=1)
print(augmat)
\end{python}
\vspace*{-0.7cm}
\begin{codeout}
[[ 1  2 -3  7]
 [ 3 -1  4  6]
 [-2  1  1  1]]
\end{codeout}
\gr
Σχεδιάζουμε μια συνάρτηση που επιλέγει μία γραμμή από τον πίνακά μας και επιστρέφει αυτή τη γραμμή συμπληρωμένη με μηδενικά στις διαστάσεις του πίνακα.
\en
\begin{python}
#####pickUpfunction######
def pick_up(the_array,get_row,give_row):
    the_row = the_array[get_row]
    to_return = np.zeros(the_array.shape)
    for x in range(len(the_row)):
        to_return[give_row,x] = the_row[x]
    return to_return
\end{python}
\gr 

Πάμε τώρα να εφαρμόσουμε τον αλγόριθμο {\en Gauss}. Σημειώνουμε εδώ ότι προφανώς ξεκινάμε από το $0$ την αρίθμηση των στοιχείων του $Α$ για λόγους συμβατότητας με την {\en numpy} και την {\en python}.

\en
\begin{python}
#####start the algorithm manually#####
i=0
j=1
ratio = augmat[j,i]/augmat[i,i]
augmat = augmat - ratio*pick_up(augmat,i,j)
print(augmat)

i=0
j=2
ratio = augmat[j,i]/augmat[i,i]
augmat = augmat - ratio*pick_up(augmat,i,j)
print(augmat)

i=1
j=2
ratio = augmat[j,i]/augmat[i,i]
augmat = augmat - ratio*pick_up(augmat,i,j)
print(augmat)
\end{python}
\vspace*{-0.7cm}
\begin{codeout}
[[  1.   2.  -3.   7.]
 [  0.  -7.  13. -15.]
 [ -2.   1.   1.   1.]]
[[  1.   2.  -3.   7.]
 [  0.  -7.  13. -15.]
 [  0.   5.  -5.  15.]]
[[  1.           2.          -3.           7.        ]
 [  0.          -7.          13.         -15.        ]
 [  0.           0.           4.28571429   4.28571429]]
\end{codeout}
\gr 
Τώρα πλέον έχουμε την ανηγμένη κλιμακωτή μορφή. Συνεπώς:
\en
\begin{python}
#####now substitute accordingly#######
x_3 = augmat[2,2]/augmat[2,3]

i = 1
x_2 = augmat[i,3]
j = 2
x_2 = x_2 - augmat[i,j]*x_3
x_2 = x_2/augmat[i,i]

i = 0
x_1 = augmat[i,3]
j = 1
x_1 = x_1 - augmat[i,j]*x_2
j = 2
x_1 = x_1 - augmat[i,j]*x_3
x_1 = x_1/augmat[i,i]

print(x_1,x_2,x_3)
\end{python}
\vspace*{-0.7cm}
\begin{codeout}
1.9999999999999996 4.000000000000001 1.0000000000000004
\end{codeout}
\gr
Ελέγχουμε το αποτέλεσμά μας:
\en
\begin{python}
print(np.linalg.solve(A,b[0]))
\end{python}
\vspace*{-0.7cm}
\begin{codeout}
[2. 4. 1.]
\end{codeout}
\gr